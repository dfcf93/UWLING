\documentclass[11pt]{article}
\usepackage{url}
\usepackage{pgfplots} 
\pgfplotsset{compat=newest} 
\setlength\topmargin{-0.6cm}   
\setlength\textheight{23.4cm}
\setlength\textwidth{17.0cm}
\setlength\oddsidemargin{0cm} 
\begin{document}
\title{Ling 572 Reading 2}
\author{Daniel Campos  \tt {dacampos@uw.edu}}
\date{02/19/2019}
\maketitle 
\section{ Q1: What is a hyperplane? What does each axis in the feature space represent? }
A hyper plane is a subspace in a feature space that is used to separate data. The hyper plane take a valye that is one dimension less than its feature space in order to create a diferentiable plane. Each axis in the feature space represent a feature that is used in classification ranging from word occourences to pixel symetry. 
\section{ Q2: What does SVM try to optimize?}
The SVM tries to optimize the width of the maximum margin hyperplane. In other words, the SVM is trying to make the biggest hyperplane that comes come closest to point withing the negative and possitive classification data. 
\section{ Q3:What is a kernel? What’s the benefit of using kernels?}
A kernel is a method used in machine to transform data into a higher dimension that has a clear dividing margin between classes. While one can manually transform data into a higher dimensionality this can be computationally expensive and by use of kernels SVMs can reap all the benefits by doing a inner product computation.
\section{Q4: What is soft margin? Why do we need it?}
A soft margin is a modification to the SVM method allows to build the margin hyperplane which contain examples in the data. This method is used because it is common to have training data with noisy labels or data that may be non differentiable. Its a simple way to ignore a few points within a margin in order to build the best hyperplane.
\end{document}



